\chapter{Performance Evaluation}

\section{Methodology}

\paragraph{Datasets}
Currently there are three datasets available to run cpp\_gbdt on. Abalone, Adult and YearMSD. But it is very easy to add more: There might be a small amount of manual preprocessing necessary to get a clean input file with comma separated values from a dataset. Once this is achieved, about 15 lines of code in the Parser class are required, to specify how the new dataset looks (size, target feature, feature type, regression/classification etc.). We can only do regression and binary classification though.

\paragraph{Implementations}

\begin{itemize}
    \item python\_gbdt
\end{itemize}

\begin{itemize}
    \item cpp\_gbdt
\end{itemize}

In terms of algorithm cpp\_gbdt does exactly the same as it's python equicalent. However, there are various benefits that C/C++ gives you: Compiler optimizations, threads.

\begin{itemize}
    \item enclave\_cpp\_gbdt
\end{itemize}

This shows the effect of putting running cpp\_gbdt inside a SGX enclave. That means: no multithreading, different ways to obtain randomness. More changes TODO.

\begin{itemize}
    \item hardened\_gbdt
\end{itemize}

\section{Accuracy}

Dummy text.

\section{Runtime}

Note, all runtime results describe the amount of time spent to perform 5-fold cross validation on a dataset.
Note, many hyperparameters don't have a huge influence on 


