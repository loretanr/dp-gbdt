\begin{abstract}
  Cyber risk assessment is an emerging field, in particular for insurance companies, that offer cyber insurance packages to their clients. However, understanding cyber risk is not easy, especially due to the lack of historical data. While machine learning can help, reconciling accuracy, explainability and privacy when data is lacking is challenging. Some machine learning techniques are better than others at addressing these challenges. For instance, ensemble methods such as gradient boosted decision trees (GBDT) are easily explainable, due to their tree structure. This is not the case of other methods such as deep neural networks, which are much more complex to interpret. GBDT models can also provide privacy through differential privacy. However, current state of the arts on differentially private GBDT models suffers from low accuracy when there are limited training data. In this thesis, we propose a new decision tree induction algorithm, \textit{2-nodes}, that enhances accuracy over small datasets while satisfying $\epsilon$-differential privacy. Further, we propose an algorithm based on Bayesian networks to generate synthetic data for cyber risk assessment. Finally, we evaluate our model on various real and synthetic datasets and show that our new induction method is able to improve accuracy on small datasets.
  \end{abstract}
